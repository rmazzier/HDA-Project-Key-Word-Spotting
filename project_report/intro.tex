% !TEX root = template.tex

\section{Introduction}
\label{sec:introduction}
%Intro is here. Points to remember:
%\begin{itemize}
%	\item General short intro: why is this an interesting problem?
%	\item Go on more specifically hinting at what I have done;
%	\item Paper contributions: what is the problem, relevance, approach, value, applicability.
%	\item overview of the structure of the paper..
%	\item stay coincise: try to stay inside one page/one half
	
The Keyword Spotting (KWS) task consists in the developement of systems which are able to recognize the presence of a certain predetermined set of keywords from a stream of user utterances. In recent years, this problem has become increasingly popular and, with the rapid development of mobile devices, it is now playing a vital role in human-computer interaction and encouraging the adoption of hands-free interfaces for a wide variety of use cases, that can range from \say{smart home} devices like Amazon Alexa, to virtual assistants such as Apple's Siri or Google Assistant. KWS systems play a complementary role with respect to more complex, cloud based speech recognition services: indeed, their are required to run on local devices and are often used to initiate interactions by pronouncing the correct keyword. For this reason the research in KWS typically focuses on a tradeoff between highly precise predictions and small memory/computational footprint.

At present, Deep Learning techinques represent the state of the art approach to the KWS problem and have shown to give good results on the performance/footprint tradeoff \cite{chen2014dnns,Sainath2015ConvolutionalNN}. Research on this field is active and novel architectures are continuously being proposed. The most common class of types that tackle the KWS problem are Convolutional Neural Networks (CNN), and Recurrent Neural Netoworks (RNN).

