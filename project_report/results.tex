% !TEX root = template.tex

\section{Results}
\label{sec:results}

In Table \ref{table:results} we report the different accuracies...
%
%In this section, you should provide the numerical results. You are free to decide the structure of this section. As a general ``rule of thumb'', use plots to describe your results, showing, e.g., precision, recall and \mbox{F-measure} as a function of the system (learning) parameters. You can also show the precision matrix. 
%
%\begin{remark}
%Present the material in a progressive and logical manner, starting with simple things and adding details and explaining more complex findings as you go. Also, do not try to explain/show multiple concepts within the same sentence. Try to \textbf{address one concept at a time}, explain it properly, and only then move on to the next one.
%\end{remark}
%
%\begin{remark}
%The best results are obtained by generating the graphs using a vector type file, commonly, either \texttt{encapsulated postscript (eps)} or \texttt{pdf} formats. To plot your figures, use the Latex \texttt{\textbackslash includegraphics} command. Lately, I tend to use pdf more.
%\end{remark}
%
%\begin{remark}
%If your model has hyper-parameters, show selected results for several values of these. Usually, tables are a good approach to concisely visualize the performance as hyper-parameters change. It is also good to show the results for different flavors of the learning architecture, i.e., how architectural choices affect the overall performance. An example is the use of CNN only or CNN+RNN, or using inception for CNNs, dropout for better generalization or attention models. So you may obtain different models that solve the same problem, e.g., CNN, CNN+RNN, CNN+inception, etc.
%\end{remark}
\begin{table}
	\caption{Table with results}
\begin{tabular}{lccc}
	
	\hline 
	Model & Param. & Mult. & Acc. \\
	\hline \hline
	
	res8-narrow[7] & $20 \mathrm{~K}$ & $5.65 \mathrm{M}$ & $90.1 \%$ \\
	res15-narrow[7] & $43 \mathrm{~K}$ & $160 \mathrm{M}$ & $94.0 \%$ \\
	res8[7] & $111 \mathrm{~K}$ & $30 \mathrm{M}$ & $94.1 \%$ \\
	res15[7] & $239 \mathrm{~K}$ & $894 \mathrm{M}$ & $95.8 \%$ \\
	\hline 
	DS-CNN-S[5] & $24 \mathrm{~K}$ & $5.4 \mathrm{M}$ & $94.4 \%$ \\
	DS-CNN-M[5] & $140 \mathrm{~K}$ & $19.8 \mathrm{M}$ & $94.9 \%$ \\
	DS-CNN-L[5] & $420 \mathrm{~K}$ & $56.9 \mathrm{M}$ & $95.4 \%$ \\
	\hline TC-ResNet8[8] & $66 \mathrm{~K}$ & $1.12 \mathrm{M}$ & $96.1 \%$ \\
	TC-ResNet8-1.5[8] & $145 \mathrm{~K}$ & $2.20 \mathrm{M}$ & $96.2 \%$ \\
	TC-ResNet14[8] & $137 \mathrm{~K}$ & $2.02 \mathrm{M}$ & $96.2 \%$ \\
	TC-ResNet14-1.5[8] & $305 \mathrm{~K}$ & $4.13 \mathrm{M}$ & $\mathbf{9 6 . 6 \%}$ \\
	\hline TENet6-narrow & $\mathbf{1 7 K}$ & $\mathbf{5 5 3 K}$ & $96.0 \%$ \\
	TENet12-narrow & $31 \mathrm{~K}$ & $895 \mathrm{~K}$ & $96.3 \%$ \\
	TENet6 & $54 \mathrm{~K}$ & $1.68 \mathrm{M}$ & $96.4 \%$ \\
	TENet12 & $100 \mathrm{~K}$ & $2.90 \mathrm{M}$ & $\mathbf{9 6 . 6 \%}$ \\
	\hline
	
	\label{table:results}
\end{tabular}
\end{table}