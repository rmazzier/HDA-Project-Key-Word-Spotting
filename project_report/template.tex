\documentclass[10pt, conference, letterpaper]{IEEEtran}

\usepackage{algorithm}
\usepackage{algorithmicx}
\usepackage{algpseudocode}
\usepackage{amsfonts}
\usepackage{amsmath}
\usepackage{amssymb}
\usepackage[ansinew]{inputenc} 
\usepackage{xcolor}
\usepackage{mathtools}
\usepackage{graphicx}
\usepackage{caption}
\usepackage{subcaption}
\usepackage{import}
\usepackage{multirow}
\usepackage{cite}
\usepackage[export]{adjustbox}
\usepackage{breqn}
\usepackage{mathrsfs}
\usepackage{acronym}
%\usepackage[keeplastbox]{flushend}
\usepackage{setspace}
\usepackage{bm}
\usepackage{stackengine}
\usepackage{dirtytalk}
\usepackage{tipa}
\usepackage{pifont}
\usepackage{listings}
\usepackage[numbers]{natbib}
\usepackage{hyperref}

\lstset{%
 backgroundcolor=\color[gray]{.85},
 basicstyle=\small\ttfamily,
 breaklines = true,
 keywordstyle=\color{red!75},
 columns=fullflexible,
}%

\lstdefinelanguage{BibTeX}
  {keywords={%
      @article,@book,@collectedbook,@conference,@electronic,@ieeetranbstctl,%
      @inbook,@incollectedbook,@incollection,@injournal,@inproceedings,%
      @manual,@mastersthesis,@misc,@patent,@periodical,@phdthesis,@preamble,%
      @proceedings,@standard,@string,@techreport,@unpublished%
      },
   comment=[l][\itshape]{@comment},
   sensitive=false,
  }

\usepackage{listings}

% listings settings from classicthesis package by
% Andr\'{e} Miede
\lstset{language=[LaTeX]Tex,%C++,
    keywordstyle=\color{RoyalBlue},%\bfseries,
    basicstyle=\small\ttfamily,
    %identifierstyle=\color{NavyBlue},
    commentstyle=\color{Green}\ttfamily,
    stringstyle=\rmfamily,
    numbers=none,%left,%
    numberstyle=\scriptsize,%\tiny
    stepnumber=5,
    numbersep=8pt,
    showstringspaces=false,
    breaklines=true,
    frameround=ftff,
    frame=single
    %frame=L
}

\renewcommand{\thetable}{\arabic{table}}
\renewcommand{\thesubtable}{\alph{subtable}}

\DeclareMathOperator*{\argmin}{arg\,min}
\DeclareMathOperator*{\argmax}{arg\,max}

\def\delequal{\mathrel{\ensurestackMath{\stackon[1pt]{=}{\scriptscriptstyle\Delta}}}}

\graphicspath{{./figures/}}
\setlength{\belowcaptionskip}{0mm}
\setlength{\textfloatsep}{8pt}

\newcommand{\eq}[1]{Eq.~\eqref{#1}}
\newcommand{\fig}[1]{Fig.~\ref{#1}}
\newcommand{\tab}[1]{Tab.~\ref{#1}}
\newcommand{\secref}[1]{Section~\ref{#1}}

\newcommand\MR[1]{\textcolor{blue}{#1}}
\newcommand\red[1]{\textcolor{red}{#1}}
\newcommand{\mytexttilde}{{\raise.17ex\hbox{$\scriptstyle\mathtt{\sim}$}}}

%\renewcommand{\baselinestretch}{0.98}
% \renewcommand{\bottomfraction}{0.8}
% \setlength{\abovecaptionskip}{0pt}
\setlength{\columnsep}{0.2in}

% \IEEEoverridecommandlockouts\IEEEpubid{\makebox[\columnwidth]{PUT COPYRIGHT NOTICE HERE \hfill} \hspace{\columnsep}\makebox[\columnwidth]{ }} 

\title{Attention Based Models for Keyword Spotting\\}

\author{Riccardo Mazzieri
%	$^\dag$
%\thanks{$^\dag$Department of Information Engineering, University of Padova, email: \{rossi\}@dei.unipd.it}
} 

\IEEEoverridecommandlockouts

\newcounter{remark}[section]
\newenvironment{remark}[1][]{\refstepcounter{remark}\par\medskip
   \textbf{Remark~\thesection.\theremark. #1} \rmfamily}{\medskip}

\begin{document}

\maketitle

\begin{abstract}
	%TODO
In recent years, Keyword Spotting (KWS) systems can be found in almost any device, ranging from smartphones, smart home devices or modern cars. They require real-time interaction and a high accuracy in order to function smoothly, even in heavily resource-costrained devices: for this reason research in KWS tries to engineer systems able to provide a good balance in the performance/lightness tradeoff.
In this work we explore a variety of deep neural newtork architectures, with particular focus on the attention mechanism, for the Keyword Spotting task (KWS). Indeed, in recent years attention based models have proven to be successful in a wide variety of domains, as well as being inherently interpretable. Motivated by this trend, we propose a variety of attention based architectures for keyword spotting, taking the Att-RNN model introduced by De Andreade et al. \cite{attention2018andreade} as a baseline. We find that models based on multi head attention layers perform slightly better, even if this comes at the cost of an increased memory footprint. 
\end{abstract}

\IEEEkeywords
Keyword Spotting, Convolutional Neural Networks, Recurrent Neural Networks, Attention Mechanism, Transformers. 
%\MR{A list of keywords defining the tools and the scenario. I would not go beyond {\it six} keywords.}
\endIEEEkeywords


\input{Intro}

% !TEX root = template.tex

\section{Related Work}
\label{sec:related_work}

\subsection{Foundations and state of the art}

In recent years, machine learning techniques have proven to be the de-facto standard for approaching the KWS problem. Such models typically perform segmentation of the audio sequence on the time domain and extract log-mel scale spectrograms or mel-frequency cepstral coefficients (MFCC) \cite{mfccs1980davis} from each frame. Those are then used as input feature vectors for the models.
One of the first works exploring deep neural networks for the KWS task is from Chen et al \cite{dnns2014chen}: the authors explore small footprint fully connected architectures and show how they improved performances with respect to the baseline HMM models. This kind of architecture accounted for the sequentiality of audio data by stacking feature vectors of adjacent audio frames: while this gives good results compared to the baseline, it is a very simplistic way to model sequential data. Sainath and Parada~\cite{convnns2015sainath} approached the problem by using CNNs, a more fitting class of models which are able, by design, to capture several essential features of speech data (like input topology or translational invariance) and at the same time having a much smaller memory footprint due to parameter sharing. Indeed, CNNs have proven to be very successful for KWS: the current state of the art has recently been achieved in \cite{kim2021broadcasted}, where the authors introduce Broadcasted Residual Networks (BC-ResNets), a particular class of ResNets that are able to capture both 1D and 2D features thanks to the introduction of a new kind of residual block.

\subsection{Models based on Attention}
As a premise, it is useful to note that in this applications of attention, the query, key and value vectors are all represented by the feature vectors from the input sequence; therefore when referring to the attention mechanism, we will actually implicitly mean self-attention \cite{selfatt2016cheng}. Specifically, dot-product attention is adopted \cite{luong2015effective}.
In \cite{attention2018andreade}, De Andreade et al. propose an attention based convolutional recurrent neural network (Att-RNN), which takes as input a mel-scale spectrogram and extracts features in the frequency dimension with two convolutional layers. Then, such features are given as input to two bidirectional LSTM \cite{hochreiter1997long} layers, in order to extract long range dependencies from the inherently sequential audio data. The last\footnote{In the paper, the authors report to use the middle output, but looking at their implementation, they use the last one. Either way, as they point out, any output of the LSTM layer should work well as a query vector, since the bidirectional LSTM layer should be able to summarize the whole sequence in any of its outputs.} output feature from the LSTM layer is then transformed with a linear dense layer and used as a query vector for the attention mechanism. Finally, the sum of the LSTM outputs, weighted with respect to the attention scores, are given as input to a final dense MLP, which performs the classification. An immediate drawback of this approach is its impossibility to function in a streaming fashion (at least without introducing delay), since bidirectional recurrent layers require the whole sequence of data to work. Furthermore, by using a single vector representation as a query vector representative for the whole sequence, this could act as an information bottleneck. Nevertheless, the introduction of the attention mechanisms allows us to inspect the sections of audio which were responsible for the classification, increasing model transparency. In \cite{streamingkws2020Rybakov} the authors propose a more modern variant of the same architecture (referred as MHAtt-RNN): LSTM is substitued by GRU \cite{Cho2014gru}, and the attention mechanism is replaced by a ~\mbox{multi-head} attention layer using 4 heads. Despite increasing accuracy with respect to Att-RNN, this kind of models can be very heavy in terms of memory footprint, especially if using a high encoding dimension for the multi-head attention layer. In the case of \cite{streamingkws2020Rybakov}, the number of parameters is $743000$ which is more than four times the number of parameters of Att-RNN.

The latest contribution in attention based models for KWS comes from \cite{kwtransformer2021berg}, where the authors, inspired by the success of the newly introduced Vision Transformer (ViT) \cite{vit2020Dosovitskiy} for computer vision tasks, propose an adaptation of such architecture for keyword spotting, called the Keyword Transformer (KWT). The results were suprising: despite ViT proved to be competitive only when supported by pre-training on large datasets, KWT outperformed more complex models based on mixes between CNN, RNN and attention, even when trained on relatively small datasets, like the Google Speech Commands dataset \cite{speechdataset2018warden}. Even in this case, despite good performances, it might not be feasible to integrate the KWT model on small mobile devices: indeed, the best performing variation of the KWT  that was proposed is relying on a very deep architecture (12 Transformer encoder layers), which counts more than 5 million parameters.

By contrast, we propose lighter attention based architectures, taking Att-RNN as the baseline.


% !TEX root = template.tex
\section{Experimental Setup}

\subsection{Signals and Features}
\label{sec:sig&features}

%Com'è , il dataset, che tipo di features uso, spiegare come sono estratte, spiegare augmentation strategies, spiegare come genero gli split del dataset per ogni task.

Each model was trained on the Google Speech Commands dataset V2 \cite{speechdataset2018warden}, which consists on a total of $105829$ user utterances of a total of $35$ keywords. Each utterance is stored as a one second long\footnote{Some clips are a bit shorter than one second: in those cases, the clips are zero padded towards the end.} \verb|WAVE| file, sampled at a 16KHz rate. From each signal, $80$ log-mel spectrogram features are computed, %TODO: non ancora sicuro se uso i mfcc o logmel spec...
using a window size of $25$\textit{ms} ($400$ samples) and a hop size of $10$\textit{ms} ($160$ samples). Following the approach from past literature, the models are trained for two different tasks, which are the following:
\begin{itemize}
	\item \textbf{12kws} task: the model must discriminate among 12 different keywords: \say{yes}, \say{no}, \say{up}, \say{down}, \say{left}, \say{right}, \say{on}, \say{off}, \say{stop}, \say{go}, unknown or silence. The unknown keywords are randomly chosen from the set of remaining keywords and the silence samples consist in one second long crops, randomly extracted from the noise files provided by the dataset\footnote{Those consist in 6 files containing noisy background sounds, both artificially generated and recorded from real environments.}. Since the total amount of words belonging to the \say{unknown} class was much more than number of representatives for each of the other keywords, we decided to randomly extract them to create a balanced dataset. In this way, we have 36921 samples for training, 4443 for validation and 4888 for testing.
	\item \textbf{35kws} task: the model must discriminate among all the 35 keywords present in the dataset. No unknown class or silence class is introduced, therefore, for this task, the entire dataset is used. The training set consists in 84843 samples, the validation in 9981 and the test in 11005. 
\end{itemize}

Following Google's suggestions from \cite{speechdataset2018warden}, the dataset is split in 80\% for training set, 10\% for validation set and 10\% for test set, in such a way that the same speakers are never present in two different splits. 

For each task, the training set is augmented, following existing approaches from the literature. Specifically, the augmentation process follows this order:
\begin{enumerate}
	\item Each signal is randomly shifted left or right (zero padding the remaining portion), by $x$ samples, where $x$ is drawn from a uniform distirbution $U(0,100)$;
	\item Samples belonging to the \say{silence} class do not remain the same across epochs, but are randomly generated each time with the same process described above;
	\item With a probability of $0.8$, each signal is mixed with a randomly generated background noise, multiplied by a factor drawn from $U(0,0.2)$; 
	\item After the conversion in MFCC/log-mel %TODO stessa roba di sopra
		the features are augmented using SpecAugment \cite{park2019specaugment}, a simple data augmentation method which consists in adding randomly sized frequency and time masks to the feature matrix: this is done in order to render the model more robust to partial loss of frequency information or of small sections of speech.
\end{enumerate}

The image resulting from the last step constitutes the actual input for the model. We remark that validation and test sets are not touched by this process.

All the project was carried out using an NVIDIA GeForce GTX 1060 6GB GPU and an Intel i5-64000 CPU, on a Linux machine. All models were built and trained using TensorFlow~\cite{Abadi2016TensorFlowAS}. All source code is available on Github\footnote{Link here.}. %TODO mettere link github

\subsection{Input Pipeline}
\label{sec:processing_architecture}
%\textbf{Why having this section:} With this section, we start the technical description with a {\it high level} introduction of your work (e.g., processing pipeline). Here, you do not have to necessarily go into the technical details of every block/algorithm of your design, this will be done later as the paper develops. What I would like to see here is a high level description of the approach, i.e., which processing blocks you used, what they do (in words) and how these were combined, etc. This section should introduce the reader to your design, explain the different parts/blocks, how they interact and why. You should not delve into technical details for each block, but you should rather explain the big picture. \MR{Besides a well written explanation, I often use a nice diagram containing the various blocks and detailing their interrelations.}
Since data augmentation is applied, storing the entire dataset in memory would not be possible: this would result in the same data being reused across epochs. For this reason, a core part of this work was to build an efficient input pipeline that could handle data augmentation on the fly, during training. In this section we explain how the data generation process described in Section \ref{sec:sig&features} works. The augmentation takes place in two different moments during training: 
\begin{enumerate}
	\item Phase 1: this is the phase when random noise samples are extracted, and each sample is shifted. This is performed by the CPU, while the GPU is performing training. 
	\item Phase 2: this phase is performed inside the model by the GPU, with a series of preprocessing layers. Specifically, several custom preprocessing layers were built:
	%TODO add descriptions for each of them e default hyperparameters!
	\begin{enumerate}
		\item \verb|RandomNoiseAugment|
		\item \verb|LogMelSpectrogram|
		\item \verb|MFCC|
		\item \verb|SpecAugment|
	\end{enumerate}
\end{enumerate}

Note that in Phase 1, each operation is performed one sample at a time, while in Phase 2 each operation is performed in parallel on an entire batch of samples: this results in a much more optimized implementation. Given the hardware setup with which the project was carried on, this framework was a necessity: just computing the MFCC for each sample on CPU would cause a performance bottleneck, wasting the computational capabilities of the GPU. In Figure \ref{fig:inputpipeline} we report a detailed diagram, showing all the steps in the input pipeline for the generation of the training set.

\begin{figure}[h]
	\centering
	\includegraphics[width=0.99\linewidth]{imgs/input_pipeline_v3.pdf}
	%TODO: image is still wip
	\caption{Detailed description of the system that generates the training data. In the cache block, the data that was produced until that moment is saved to file. Each step which is performed before the caching operation happens only one time, during the first iteration of the dataset; on all the successive iterations, the data is read from the cached file. For the validation and test sets, the augmentation is not applied. Furthermore, to ensure complete separation between training and validation/test sets, the noise samples were generated from different portions of the noise files based on whether they were used for training of for validation or testing.}
	\label{fig:inputpipeline}
\end{figure}





\subsection{Learning Framework}
\label{sec:learning_framework}

Descrivere nel dettaglio i modelli
[...]

Each model was trained for 50 epochs, using early stopping and reduce lr on plateau %TODO: spiegare bene

%Here you finally describe the learning strategy / algorithm that you conceived and used to solve the problem at stake. A good diagram to exemplify how learning is carried out is often very useful. In this section, you should describe the learning model, its parameters, any optimization over a given parameter set, etc. You can organize this section into \mbox{sub-sections}. You are free to choose the most appropriate structure.
%
%\begin{remark}
%Note that the diagram that you put here differs from that of Section~\ref{sec:processing_architecture} as here you show the details of how your learning framework, or the core of it, is built. In Section~\ref{sec:processing_architecture} you instead provide a high-level description of the involved processing blocks, i.e., you describe the {\it processing flow} and the rationale behind it.
%\end{remark}
%
%
%


% !TEX root = template.tex

\section{Results}
\label{sec:results}

In Table \ref{table:results} we report the different accuracies...
%
%In this section, you should provide the numerical results. You are free to decide the structure of this section. As a general ``rule of thumb'', use plots to describe your results, showing, e.g., precision, recall and \mbox{F-measure} as a function of the system (learning) parameters. You can also show the precision matrix. 
%
%\begin{remark}
%Present the material in a progressive and logical manner, starting with simple things and adding details and explaining more complex findings as you go. Also, do not try to explain/show multiple concepts within the same sentence. Try to \textbf{address one concept at a time}, explain it properly, and only then move on to the next one.
%\end{remark}
%
%\begin{remark}
%The best results are obtained by generating the graphs using a vector type file, commonly, either \texttt{encapsulated postscript (eps)} or \texttt{pdf} formats. To plot your figures, use the Latex \texttt{\textbackslash includegraphics} command. Lately, I tend to use pdf more.
%\end{remark}
%
%\begin{remark}
%If your model has hyper-parameters, show selected results for several values of these. Usually, tables are a good approach to concisely visualize the performance as hyper-parameters change. It is also good to show the results for different flavors of the learning architecture, i.e., how architectural choices affect the overall performance. An example is the use of CNN only or CNN+RNN, or using inception for CNNs, dropout for better generalization or attention models. So you may obtain different models that solve the same problem, e.g., CNN, CNN+RNN, CNN+inception, etc.
%\end{remark}

In Figure \ref{fig:accs_vs_parameters} we plot the test set accuracy with their number of parameters, both for the 12KWS and 35KWS tasks.

\begin{figure*}
	\centering
	\begin{subfigure}{.5\textwidth}
		\centering
		\includegraphics[width=\linewidth]{imgs/size_vs_accuracy12.pdf}
		\caption{A subfigure}
		\label{fig:sub1}
	\end{subfigure}%
	\begin{subfigure}{.5\textwidth}
		\centering
		\includegraphics[width=\linewidth]{imgs/size_vs_accuracy35.pdf}
		\caption{A subfigure}
		\label{fig:sub2}
	\end{subfigure}
	\caption{A figure with two subfigures}
	\label{fig:accs_vs_parameters}
\end{figure*}

\begin{table}
	\caption{Table with results}
\begin{tabular}{lccc}
	
	\hline 
	Model & Param. & Mult. & Acc. \\
	\hline \hline
	
	res8-narrow[7] & $20 \mathrm{~K}$ & $5.65 \mathrm{M}$ & $90.1 \%$ \\
	res15-narrow[7] & $43 \mathrm{~K}$ & $160 \mathrm{M}$ & $94.0 \%$ \\
	res8[7] & $111 \mathrm{~K}$ & $30 \mathrm{M}$ & $94.1 \%$ \\
	res15[7] & $239 \mathrm{~K}$ & $894 \mathrm{M}$ & $95.8 \%$ \\
	\hline 
	DS-CNN-S[5] & $24 \mathrm{~K}$ & $5.4 \mathrm{M}$ & $94.4 \%$ \\
	DS-CNN-M[5] & $140 \mathrm{~K}$ & $19.8 \mathrm{M}$ & $94.9 \%$ \\
	DS-CNN-L[5] & $420 \mathrm{~K}$ & $56.9 \mathrm{M}$ & $95.4 \%$ \\
	\hline TC-ResNet8[8] & $66 \mathrm{~K}$ & $1.12 \mathrm{M}$ & $96.1 \%$ \\
	TC-ResNet8-1.5[8] & $145 \mathrm{~K}$ & $2.20 \mathrm{M}$ & $96.2 \%$ \\
	TC-ResNet14[8] & $137 \mathrm{~K}$ & $2.02 \mathrm{M}$ & $96.2 \%$ \\
	TC-ResNet14-1.5[8] & $305 \mathrm{~K}$ & $4.13 \mathrm{M}$ & $\mathbf{9 6 . 6 \%}$ \\
	\hline TENet6-narrow & $\mathbf{1 7 K}$ & $\mathbf{5 5 3 K}$ & $96.0 \%$ \\
	TENet12-narrow & $31 \mathrm{~K}$ & $895 \mathrm{~K}$ & $96.3 \%$ \\
	TENet6 & $54 \mathrm{~K}$ & $1.68 \mathrm{M}$ & $96.4 \%$ \\
	TENet12 & $100 \mathrm{~K}$ & $2.90 \mathrm{M}$ & $\mathbf{9 6 . 6 \%}$ \\
	\hline
	
	\label{table:results}
\end{tabular}
\end{table}

\subsection{Attention Plots}
We can plot the attention scores for our models to see which portions of the audio files were catching the attention of the model... In Figure \ref{fig:att_scores} we can make a comparison between multi head and non multi head...

\begin{figure}[h!]
	\centering
	\begin{tabular}{cc}
		\includegraphics[width = 0.45\linewidth]{imgs/att_scores12_23_andreade.pdf} &
		\includegraphics[width = 0.45\linewidth]{imgs/att_scores12_23.pdf}\\
		\includegraphics[width = 0.45\linewidth]{imgs/att_scores12_42_andreade.pdf} &
		\includegraphics[width = 0.45\linewidth]{imgs/att_scores12_42.pdf}\\
		\includegraphics[width = 0.45\linewidth]{imgs/att_scores12_72_andreade.pdf} &
		\includegraphics[width = 0.45\linewidth]{imgs/att_scores12_72.pdf}
	\end{tabular}
	\caption{Comparison between attention scores from  Att-RNN model (left) and MHAtt-RNN with three heads (right), on the words \textit{off}, \textit{yes} and \textit{no}. We can see that Att-RNN has only one head, so one set of attention scores per prediction is computed. MHAtt-RNN instead computes one set of attention weights per head: here we visualize the attention scores for each head. In these examples, we can see how each head learns to pay attention to different phonemes of the same word. In the first example, Att-RNN pays attention only to the first phoneme /\textipa{o}/, while MHAtt-RNN has two heads paying attention to /o/ and one paying attention to /f/. In the second example, a similar thing happens: Att-RNN pays attention just at the /\textipa{je}/ while MHAtt-RNN has different heads concentrating both on /\textipa{je}/ and /\textipa{s}/. The third example presents a noise at the beginning which is not part of the spoken word: Att-RNN has its attention drawn a bit, while two of three heads from MHAtt-RNN learn to completely ignore it.}
	\label{fig:att_scores}
\end{figure}



% !TEX root = template.tex

\section{Concluding Remarks}

Concluding remarks bla bla bla bla bla bla bla bla bla bla bla bla bla bla bla bla bla bla bla bla bla bla bla bla bla bla bla bla bla bla bla bla bla bla bla bla  

This is not a complete way to evaluate a KWS model: one should also make streaming tests and analyzing the false reject rate... Also, for this project the models were trained just one time due to the limited computational capabilites; this fact makes the reported results not too statistically relevant. To perform a more rigorous statistical analysis, one should train the models more times and use an avarage of the test set accuracies among the runs, computing confidence intervals for the final accuracy.

In conclusion, this was a very instructive project to work on: I had the opportunity to study a lot of modern machine learning literature and to understand more complex architectures based on the attention mechanism. I also think that this report \LaTeX template was extremely useful and well done, and it will surely be a very useful tool for the future. Besides the part involving the study of various papers, in my experience the most difficult part of this work was building a working input pipeline, both for technical reasons (due to the limitations of my hardware) and for the difficulty to find information online. While it is true that the Labs were extremely useful and essential, especially on the part regarding the \verb|tf.data.Dataset| API, while working on the project I often came across errors which were really hard to debug mostly because of my unawareness of how Tensorflow really works under the hood (see for example the difference between Graph execution and Eager execution). Besides this aspect, I think that the course gave me strong knowledge foundations in order to build a project like this.

\label{sec:conclusions}
%
%\red{This section should take max half a page, I personally find it difficult to come up with really useful observations, I mean ones that bring a new contribution with respect to what you have already expounded in the ``Results'' section. In case you have some serious stuff to write, you may also extend the section to 3/4 of a page :-).}\\
%
%In many papers, here you find a summary of what done. It is basically an abstract where instead of using the present tense you use the past participle, as you refer to something that you have already developed in the previous sections. While I did it myself in the past, I now find it rather useless.\\ 
%
%\MR{\textbf{What I would like to see here is:} 
%\begin{enumerate}
%\item a very short summary of what done, 
%\item some (possibly) intelligent observations on the relevance and {\it applicability} of your algorithms / findings, 
%\item what is still missing, and can be added in the future to extend your work.\\
%\end{enumerate}
%The idea is that this section should be {\it useful} and not just a repetition of the abstract (just \mbox{re-phrased} and written using a different tense...).}\\
%
%\red{\textbf{Moreover:} being a project report, I would also like to see a specific paragraph stating 
%\begin{enumerate}
%\item[4)] what you have learned, and 
%\item[5)] any difficulties you may have encountered.
%\end{enumerate}}


\bibliography{biblio}
\bibliographystyle{ieeetr}

\end{document}


