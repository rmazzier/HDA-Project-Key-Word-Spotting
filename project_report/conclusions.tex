% !TEX root = template.tex

\section{Concluding Remarks}

Concluding remarks bla bla bla bla bla bla bla bla bla bla bla bla bla bla bla bla bla bla bla bla bla bla bla bla bla bla bla bla bla bla bla bla bla bla bla bla  

\label{sec:conclusions}
%
%\red{This section should take max half a page, I personally find it difficult to come up with really useful observations, I mean ones that bring a new contribution with respect to what you have already expounded in the ``Results'' section. In case you have some serious stuff to write, you may also extend the section to 3/4 of a page :-).}\\
%
%In many papers, here you find a summary of what done. It is basically an abstract where instead of using the present tense you use the past participle, as you refer to something that you have already developed in the previous sections. While I did it myself in the past, I now find it rather useless.\\ 
%
%\MR{\textbf{What I would like to see here is:} 
%\begin{enumerate}
%\item a very short summary of what done, 
%\item some (possibly) intelligent observations on the relevance and {\it applicability} of your algorithms / findings, 
%\item what is still missing, and can be added in the future to extend your work.\\
%\end{enumerate}
%The idea is that this section should be {\it useful} and not just a repetition of the abstract (just \mbox{re-phrased} and written using a different tense...).}\\
%
%\red{\textbf{Moreover:} being a project report, I would also like to see a specific paragraph stating 
%\begin{enumerate}
%\item[4)] what you have learned, and 
%\item[5)] any difficulties you may have encountered.
%\end{enumerate}}
