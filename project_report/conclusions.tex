% !TEX root = template.tex

\section{Concluding Remarks}

Concluding remarks bla bla bla bla bla bla bla bla bla bla bla bla bla bla bla bla bla bla bla bla bla bla bla bla bla bla bla bla bla bla bla bla bla bla bla bla  

This is not a complete way to evaluate a KWS model: one should also make streaming tests and analyzing the false reject rate... Also, for this project the models were trained just one time due to the limited computational capabilites; this fact makes the reported results not too statistically relevant. To perform a more rigorous statistical analysis, one should train the models more times and use an avarage of the test set accuracies among the runs, computing confidence intervals for the final accuracy.

In conclusion, this was a very instructive project to work on: I had the opportunity to study a lot of modern machine learning literature and to understand more complex architectures based on the attention mechanism. I also think that this report \LaTeX template was extremely useful and well done, and it will surely be a very useful tool for the future. Besides the part involving the study of various papers, in my experience the most difficult part of this work was building a working input pipeline, both for technical reasons (due to the limitations of my hardware) and for the difficulty to find information online. While it is true that the Labs were extremely useful and essential, especially on the part regarding the \verb|tf.data.Dataset| API, while working on the project I often came across errors which were really hard to debug mostly because of my unawareness of how Tensorflow really works under the hood (see for example the difference between Graph execution and Eager execution). Besides this aspect, I think that the course gave me strong knowledge foundations in order to build a project like this.

\label{sec:conclusions}
%
%\red{This section should take max half a page, I personally find it difficult to come up with really useful observations, I mean ones that bring a new contribution with respect to what you have already expounded in the ``Results'' section. In case you have some serious stuff to write, you may also extend the section to 3/4 of a page :-).}\\
%
%In many papers, here you find a summary of what done. It is basically an abstract where instead of using the present tense you use the past participle, as you refer to something that you have already developed in the previous sections. While I did it myself in the past, I now find it rather useless.\\ 
%
%\MR{\textbf{What I would like to see here is:} 
%\begin{enumerate}
%\item a very short summary of what done, 
%\item some (possibly) intelligent observations on the relevance and {\it applicability} of your algorithms / findings, 
%\item what is still missing, and can be added in the future to extend your work.\\
%\end{enumerate}
%The idea is that this section should be {\it useful} and not just a repetition of the abstract (just \mbox{re-phrased} and written using a different tense...).}\\
%
%\red{\textbf{Moreover:} being a project report, I would also like to see a specific paragraph stating 
%\begin{enumerate}
%\item[4)] what you have learned, and 
%\item[5)] any difficulties you may have encountered.
%\end{enumerate}}
